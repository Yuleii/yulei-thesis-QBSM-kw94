\section{Introduction}
\thispagestyle{plain} % surpress header on first page

Structural econometric models have been widely used to provide decision support and guidelines for policymakers. The Discrete Choice Dynamic Programming (DCDP) model, a structural model commonly used in the field of labor economics, plays a key role in evaluating the effects of various policy interventions. In particular, DCDP models study schooling and occupational choice and investigate the effects of policy interventions designed to boost human capital investment, such as parental and government subsidies and student loan programs \citep{keane1997CareerDecisionsYoung, sauer2004EducationalFinancingLifetime}.\\

\noindent
A typical structural model of optimal individual behavior consists of a number of structural parameters that capture the underlying preferences and constraints of the individual’s decision process \citep{blundell2017WhatHaveWe}. Such models require individuals to solve a sequential decision problem under given institutional constraints that are determined by a set of variables determined by policymakers (e.g., tax rates). So far, most structural models have been deterministic, with each model parameter fixed with a single numerical value or point estimate, ignoring their associated errors or uncertainties. \citep{adda2017CareerCostsChildren, attanasio2012EducationChoicesMexico, eckstein2019CareerFamilyDecisions}. In addition to using point estimates, policymakers can obtain more information about prediction errors by including uncertainty in the model parameters. For example, using confidence sets as set estimates \citep{manski2021EconometricsDecisionMaking}. More recently, noticing the considerable parametric uncertainty in prediction, \cite{eisenhauer2021StructuralModelsPolicymaking} suggested the use of global sensitivity analysis to identify which parameters are particularly responsive to prediction uncertainty.\\

\noindent
Sensitivity analysis(SA) has been a well-established approach in a wide variety of scientific disciplines, such as biology \citep{zi2011SensitivityAnalysisApproaches}, chemistry \citep{saltelli2005SensitivityAnalysisChemical}, environmental sciences \citep{campolongo1997SensitivityAnalysisEnvironmental} and so on. However, economists have had a very restricted contribution to the literature on sensitivity analysis. Although the importance of sensitivity analysis for quantitative models has been well acknowledged by some economists, very few studies have included any form of sensitivity analysis when building economic models, or if they do, they often rely on local techniques, in which case the sensitivity measures are only computed around a fixed point in the parameter space \citep{canova1994StatisticalInferenceCalibrated, harenberg2019UncertaintyQuantificationGlobal, leamer1985SensitivityAnalysesWould}. It is notable that SA in economics research is distinguished from other areas such as environmental science by the conditional distribution of the input parameters. As a result, using a one-factor-at-a-time Local SA without accounting for any parameter correlations would yield misleading results. Moreover, Local methods have the glaring drawback that they are invalid for nonlinear and nonmonotonic models \citep{saltelli2004SensitivityAnalysisPractice}, which are frequently used in econometric modeling. In contrast, global methods with a “model-free” setting are strongly recommended in systematic reviews \citep{saltelli2019WhyManyPublished}. Unfortunately, a generally accepted practice of GSA has not yet been established in most of the existing economic literature.\\

\noindent
I rely on quantile-based sensitivity measures(QBSM) \citep{kucherenko2019QuantileBasedGlobal} to analyze the impact of the parametric uncertainty on prediction uncertainty. Such measures are based on quantiles of the model output, which allow us to identify the parametric uncertainty at quantile level. The quantiles of output Cumulative distribution(CDF) are also of interest in other domains such as reliability analysis and financial risk analysis. For such problems, conventional method such as variance-based sensitivity indices \citep{sobol1993SensitivityEstimatesNonlinear} are in practical. Moreover, variance-based sensitivity measures are not able to identify variables that are most essential in achieving the extreme values of the model output. Alternatively, QBSM are designed for global sensitivity analysis of problems where the $\alpha$-th quantile is a function of interest, and also for instances where the analyst is interested in the ranking of the inputs that cause the extreme values of the output. The quantile oriented sensitivity indices \citep{kala2019QuantileorientedGlobalSensitivity} share some similarities with QBSM, but these indices are based on contrast function instead of explicitly depending on quantiles.\\

\noindent
As an application, I reanalyze the human capital investment decision model by \cite{keane1994SolutionEstimationDiscrete}. I estimate the model parameters and replicated the key results using the provided dataset. I generate counterfactual predictions based on the estimated parameters.  I investigate the transition of uncertainty from key parameters to model output. I calculate the QBSM for a few parameters to identify the parameters that contribute most to the prediction uncertainty.  I extend their framework to explain which parameter specifically contributes to the uncertainty. \\

\noindent
This work complements the work by \cite{eisenhauer2021StructuralModelsPolicymaking}, who documented significant prediction uncertainty in a Eckstein-Keane-Wolpin(EKW) model \citep{aguirregabiria2010DynamicDiscreteChoice}. My work builds on their work by identifying which parameters account for most of the output uncertainty based on the quaniles of output. In their study, a 90\% confidence set is used to represent the counterfactual uncertainty. As a result, the analyst is primarily concerned with the region of the output values range inside the confidence set. In other words, the critical quantiles of the cumulative distribution function (CDF) of the function are of interest, which fits well with the target problem that QBSM aim to addresses. Although including sensitivity analysis in model calibration has long been a standard technique in environmental and engineering modeling, there is a lack of application in econometric modelling.\\


\noindent
This paper is organized as follows: Section \ref{sec:2} presents the model setting of \cite{keane1994SolutionEstimationDiscrete}. Section \ref{sec:3} gives an introduction to variance-based global sensitivity indices and QGSM. Algorithms and methods for numerical estimation of QGSM are discussed in Section \ref{sec:3}. The results are presented in Section \ref{sec:5}.  Finally, conclusions are summarized in Section \ref{sec:6}.\\