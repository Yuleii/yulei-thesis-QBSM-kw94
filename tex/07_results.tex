\section{Results} \label{sec:5}
\thispagestyle{plain} % surpress header on first page


The study is greatly governed by computer problems. On the one hand, DCDP models are computationally demanding, which is determined by their structure\citep{keane2011ChapterStructuralEstimation}. On the other hand, QBSM has no closed form expression to solve, thus they can only be numerically solved using MC simulation. For example, even DLR method is used, the number of function evaluations for computing QGSM for with $d$ inputs are $ N = N(dM + 1)$. To simplies the analyis, Figure \ref{fig:5} displays the estimate results of QBSM with 30 individuals and 3 chosen parameters. Even so, it is still computationally burdensome. However our aim with this quantile based sensitivity analysis of the structural econometric model is not to perform an in-depth analysis of the results, but rather to show how these statistical tools provide a better understanding of model complexity. \\

\noindent
 Table \ref{fig:6} displays QBSM of three carefully picked parameters, and through these poorly converged results we can still conclude that the parametric uncertainty in model prediction in previous literature has been overlooked. The red lines indicate the normalized QBSM of return to an additional year of schooling$\alpha_{11}$, while the blue lines indicate the normalized QBSM of return for collage education $\beta_1$. The red lines indicate the normalized QBSM of home rewards $\gamma_0$. $Q_{i}^{(1)}$ and  $Q_{i}^{(2)}$ corresponds to the measure listed in \eqref{eq:16} and \eqref{eq:17}, respectively. PDF of the output Y is symmetric(Figure \ref{fig:5}) lead to a symmetric behavior of $Q_i$ around $\alpha =0.5$. The general ranking of variables determined by QBSM is: $\alpha_{11}$, $\beta_1$, $\gamma_0$. Measures $Q_i$ approximately reaches the minimum for $\beta_1$ and $\gamma_0$ and and maximum for $\alpha_{11}$ at $\alpha=0.6$. When the quantities level $\alpha$ is very small or large, a larger sample size will required to achieve acceptable accuracy. However, the performance of these parameters around $\alpha = 0.5$ is quite stable, with $\alpha_{11}$ account for the greatest variability and $\gamma_0$ is nearly have no influence on the model uncertainty.